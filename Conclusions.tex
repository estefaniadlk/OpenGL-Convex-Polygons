\addcontentsline{toc}{chapter}{Συμπεράσματα}
\chapter*{Συμπεράσματα}
Η δυνατότητα απεικόνισης και αλληλεπίδρασης με γραφικά στοιχεία επιτρέπει στο χρήστη να έχει έναν οπτικό τρόπο αναπαράστασης και επεξεργασίας γεωμετρικών δομών. Σε συνολικό βαθμό ο κώδικας παρέχει μια ικανοποιητική και ευέλικτη βάση για την ανάπτυξη περαιτέρω λειτουργικοτήτων και εξέλιξη του προγράμματος. Προσφέρει ευελιξία και δυνατότητες για επέκταση και προσαρμογή. Με την υπάρχουσα υλοποίηση, μπορεί να γίνει αναπαραγωγή και αποθήκευση των κυρτών πολυγώνων, καθώς και προσθήκη επιπλέον λειτουργιών όπως η επεξεργασία και η μετακίνηση των πολυγώνων. Συνοψίζοντας, η εργασία αυτή εστίασε στην υλοποίηση ενός κώδικα για τη χάραξη αυθαίρετων κυρτών πολυγώνων με χρήση περιβάλλοντος γραφικών. Ο κώδικας επιτρέπει τη δημιουργία πολυγώνων με την επιλογή κορυφαίων σημείων μέσω του ποντικιού και προσφέρει την απεικόνιση των πολυγώνων στο γραφικό παράθυρο επιτυχώς. Η αλληλεπίδραση με το ποντίκι είναι ομαλή. 

\addcontentsline{toc}{chapter}{Επίλογος}
\chapter*{Επίλογος}
Μέσω αυτής της εργασίας αποκτήθηκε η κατανόηση της σχεδίασης πολυγώνων και αρχέγονων σχημάτων δύο διαστάσεων σε γραφικό περιβάλλον. Συνολικά αυτή η εργασία μας έδωσε την ευκαιρία να εξοικειωθούμε με την ανάπτυξη εφαρμογών υπολογιστικής γεωμετρίας. Με αυτήν τη βάση μπορούμε να προχωρήσουμε σε περαιτέρω αναπτύξεις και να εφαρμόσουμε τον κώδικα σε πραγματικά προβλήματα χαραγμένων αυθαίρετων κυρτών πολυγώνων και γεωμετρικών εφαρμογών γενικότερα, θέτοντας στόχους βελτιστοποίησης. 

