\addcontentsline{toc}{chapter}{Περίληψη}
\chapter*{Περίληψη}

Στην παρούσα εργασία αναφερόμαστε στη μαθηματική θεωρία πίσω από τα κυρτά πολύγωνα, αναλύουμε μια γενικευμένη διαδικασία χάραξης αυτών από ένα δοθέντα αλγόριθμο αρχέγονου τριγώνου, και περιγράφουμε τη διαδικασία απεικόνισής τους σε υπολογιστικό περιβάλλον. Αρχικά, παρουσιάζουμε τα μαθηματικά θεμέλια που αφορούν τα κυρτά πολύγωνα, συμπεριλαμβανομένων των ιδιοτήτων τους, των γωνιών και των ακμών τους. Στη συνέχεια, εξετάζουμε τις βασικές λειτουργίες που αφορούν τα κυρτά πολύγωνα και εφαρμόζονται στη γραφική υπολογιστών, όπως ο υπολογισμός του εμβαδού και ο έλεγχος των εσωτερικών σημείων τους, καθώς και ο υπολογισμός των γραμμικών συναρτήσεων των ακμών τους για όλα τα εικονοστοιχεία του περιβάλλοντος κυτίου τους. Στο δεύτερο μέρος της εργασίας μας, επικεντρωνόμαστε στην απεικόνιση αυθαίρετων κυρτών πολυγώνων από χρήστη σε υπολογιστικό περιβάλλον χρησιμοποιώντας την \textlatin{OpenGL}. Επιπλέον, παρουσιάζουμε παραδείγματα κώδικα και εφαρμόζουμε τους αλγορίθμους ψηφιδόξυσης και υπολογισμού του εμβαδού. Τέλος, παρουσιάζουμε τα αποτελέσματα των δοκιμών που πραγματοποιήσαμε για την αξιολόγηση της απόδοσης του προτεινόμενου συστήματος. Τα αποτελέσματα δείχνουν ότι το σύστημα παρέχει αποτελεσματική και ακριβή απεικόνιση των κυρτών πολυγώνων με χρήση της βιβλιοθήκης \textlatin{OpenGL}.

\vspace{1.5em}

\section*{Λέξεις - κλειδιά}
Κυρτά πολύγωνα, ψηφιδόξυση, υπολογιστική γεωμετρία, υπολογιστική γραφική, \textlatin{OpenGL}

\newpage

\chapter*{\textlatin{Abstract}}

\textlatin{In this study, we refer to the the mathematical theory behind convex polygons, the general procedure for constructing them using a given primitive triangle algorithm, and describe the process of visualizing them in a computational environment with only mathematical operations. Initially, we present the mathematical foundations concerning convex polygons, including their properties, angles, and edges. We then examine the fundamental mathematical operations related to convex polygons, such as computing their area and convex hull, as well as calculating the linear functions of their edges for all pixel elements within their bounding boxes. In the second part of the study, we focus on the visualization of arbitrary convex polygons in a computational environment using OpenGL. Additionally, we provide code examples and implement algorithms for rasterization and area analysis. Finally, we present the results of the tests conducted to evaluate the performance of the proposed system. We assess the speed and accuracy of the computations in various scenarios and polygon sizes. The results demonstrate that the system provides efficient and accurate visualization of convex polygons using the OpenGL library.}

\vspace{1.5em}

\section*{\textlatin{Key Words}}
\textlatin{Convex polygons, rasterization, computational geometry, computer graphics, OpenGL}
